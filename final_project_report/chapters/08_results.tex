\section{Quantitative Performance}
The overall performance of both models on the test set is summarized in Table \ref{tab:results}.

\begin{table}[H]
    \centering
    \begin{tabular}{lcc}
        \toprule
        \textbf{Metric} & \textbf{Random Forest} & \textbf{Neural Network} \\
        \midrule
        Accuracy & \textbf{94\%} & 76\% \\
        Precision (Weighted) & 0.95 & 0.76 \\
        Recall (Weighted) & 0.94 & 0.76 \\
        F1-Score (Macro) & 0.94 & 0.75 \\
        \bottomrule
    \end{tabular}
    \caption{Model Performance Comparison. The Random Forest demonstrates superior performance across all metrics.}
    \label{tab:results}
\end{table}

\section{Visualizations and Interpretation}

\subsection{Confusion Matrix Analysis}
\begin{figure}[H]
    \centering
    \includegraphics[width=0.8\textwidth]{images/confusion_matrix.png}
    \caption{Confusion Matrix (Random Forest). The matrix illustrates the class-wise prediction accuracy. Rows represent true labels, columns represent predictions. The high diagnosis accuracy for Carcinoma (1.00) relative to benign conditions is a crucial safety feature for clinical deployment.}
    \label{fig:cm}
\end{figure}

The Confusion Matrix (Figure \ref{fig:cm}) provides a granular view of misclassifications:
\begin{itemize}
    \item \textbf{Carcinoma (car)}: The model achieved perfect recall. No malignant cases were missed.
    \item \textbf{Adipose (adi)}: Perfectly classified, likely due to the distinct high resistivity of fat.
    \item \textbf{Ambiguity}: Some confusion persists between Glandular (gla) and Fibro-adenoma (fad) tissues. These tissues share similar hydration levels, making electrical differentiation challenging without spatial resolution.
\end{itemize}

\subsection{Benchmarking Models}
\begin{figure}[H]
    \centering
    \includegraphics[width=0.8\textwidth]{images/model_comparison.png}
    \caption{Accuracy Comparison: Random Forest vs Neural Network. The Random Forest significantly outperforms the MLP/PINN proxy.}
    \label{fig:comparison}
\end{figure}
As shown in Figure \ref{fig:comparison}, the Random Forest outperforms the Neural Network. This is consistent with literature suggesting limited data ($N=106$) favors ensemble methods over deep learning, which is prone to overfitting in low-data regimes \cite{yang2021machine}.

\section{Physics Validation: Cole-Cole Plots}
To validate that the Neural Network learned meaningful physical representations despite its lower accuracy, we reconstructed the Cole-Cole arcs for the test samples.

\begin{figure}[H]
    \centering
    \includegraphics[width=\textwidth]{images/cole_cole_classification_NN.png}
    \caption{Cole-Cole Plot Visualization. \textbf{Left}: Ground Truth class distribution in the Resistance-Reactance plane. \textbf{Right}: Model predicted distribution. The plots show the characteristic semi-circular arcs predicted by the Cole-Cole equation. Adipose tissue (orange) forms large, high-resistance arcs, clearly separated from the smaller, low-resistance arcs of Carcinoma (green) and Glandular (red) tissue.}
    \label{fig:cole_cole}
\end{figure}

Figure \ref{fig:cole_cole} confirms the model's physics-compliance:
Figure \ref{fig:cole_cole} confirms the model's compliance with bioimpedance physics and offers crucial diagnostic inferences:

\subsubsection{Resistance Spectrum ($R$-Axis Analysis)}
The horizontal spread of the arcs represents the resistance domain. 
\begin{itemize}
    \item \textbf{High Resistance (Adipose)}: The Adipose tissue (orange arcs) extends far to the right, indicating very high $R_0$ values. This is physically consistent with fat cells having low water content and thus acting as poor conductors.
    \item \textbf{Low Resistance (Carcinoma)}: In sharp contrast, Carcinoma samples (green) are clustered tightly at the lower end of the resistance spectrum (left side). Biologically, malignant tumors often exhibit increased vascularization and a higher ratio of intracellular water, providing a path of least resistance for the current.
\end{itemize}

\subsubsection{Reactance and Membrane Integrity ($X$-Axis Analysis)}
The height of the arcs corresponds to the maximum reactance ($X_{peak}$), which is a direct proxy for membrane capacitance.
\begin{itemize}
    \item Healthy glandular tissues (red) exhibit moderate-to-high reactance, suggesting intact cell membranes that effectively store charge.
    \item The model accurately predicts the suppressed reactance in some pathological states, which aligns with the hypothesis that cancer cells often have disrupted or leaky membranes, reducing their capacitive properties.
\end{itemize}

\subsubsection{Model Fidelity}
The visual comparison between Ground Truth (Left) and Prediction (Right) reveals that the Neural Network has successfully learned the non-linear topology of the data. Despite having a lower numerical accuracy than the Random Forest on paper, the Cole-Cole visualization proves that the NN is not just guessing but has learned the underlying manifold of the tissue types. The few misclassifications (visible as wrong-colored arcs in the cluster) are primarily between Glandular and Fibro-adenoma classes, which overlap significantly even in the ground truth physics.
