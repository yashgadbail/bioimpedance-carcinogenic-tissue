\section{Overview}
Bioimpedance analysis (BIA) is a powerful, non-invasive technique used to characterize the electrical properties of biological tissues. It has gained significant attention in biomedical engineering due to its potential for low-cost, label-free diagnosis of various pathologies, including cancer. The fundamental principle of BIA relies on the fact that different tissue types---such as adipose, glandular, and malignant tissues---exhibit distinct electrical conductivities and permittivities. These differences arise from variations in cellular architecture, water content, membrane integrity, and electrolyte concentration.

In the context of breast cancer detection, BIA offers a promising adjunct to traditional screening methods like mammography and ultrasound. Mammography, while effective, involves ionizing radiation and can be uncomfortable for patients. BIA, on the other hand, uses safe, low-amplitude alternating currents to probe the tissue. When a tumor develops, the tissue structure changes drastically: cell membranes may break down, intracellular water content may increase, and neo-vascularization occurs. These physiological changes manifest as measurable alterations in the complex impedance spectrum.

This project aims to automate the classification of breast tissues by applying advanced machine learning algorithms to bioimpedance data. We utilize a dataset of Cole-Cole parameters derived from multi-frequency impedance spectroscopy. Our goal is to develop a robust classifier capable of distinguishing between healthy tissues (adipose, glandular, connective) and pathological conditions (carcinoma, fibro-adenoma, mastopathy) with high accuracy.

\section{Background}
\subsection{The Need for Non-Invasive Diagnostics}
Breast cancer remains one of the most common malignancies worldwide. Early detection is effectively the only way to reduce mortality. However, current gold-standard techniques have limitations:
\begin{itemize}
    \item \textbf{Biopsy}: Invasive, painful, and carries infection risk.
    \item \textbf{Mammography}: Uses radiation, has lower sensitivity in dense breasts, and can yield false positives leading to unnecessary biopsies.
    \item \textbf{MRI}: Highly sensitive but expensive and time-consuming.
\end{itemize}
Bioimpedance offers a solution that is radiation-free, portable, and potentially low-cost.

\subsection{Biological Basis of Impedance}
The electrical properties of tissue are determined by its microscopic structure.
\begin{itemize}
    \item \textbf{Resistance (R)}: Determined largely by the volume of extracellular and intracellular fluids. Carcinoma often has lower resistance due to increased hydration and cellularity.
    \item \textbf{Reactance ($X_c$)}: Determined by cell membranes, which act as capacitors. Healthy cells with intact membranes have high reactance. Damaged or irregular membranes in tumors may exhibit altered capacitive behavior.
\end{itemize}
By measuring these variance across frequencies, we can effectively "fingerprint" the tissue type.
