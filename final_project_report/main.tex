\documentclass[12pt,a4paper]{report}

\usepackage[utf8]{inputenc}
\usepackage[T1]{fontenc}
\usepackage{lmodern}

\usepackage{graphicx}
\usepackage{geometry}
\usepackage{float}
\usepackage{booktabs}

\usepackage{amsmath, amssymb}
\usepackage{xcolor}
\usepackage{listings}
\usepackage{cite}

\usepackage{hyperref}
\usepackage{fancyhdr}
\usepackage{titlesec}

\geometry{
    top=1in,
    bottom=1in,
    left=1.25in,
    right=1in
}

\hypersetup{
    colorlinks=true,
    linkcolor=black,
    citecolor=blue,
    urlcolor=blue
}

\pagestyle{fancy}
\fancyhf{}
\fancyhead[L]{\nouppercase{\leftmark}}
\fancyhead[R]{\thepage}
\renewcommand{\headrulewidth}{0.4pt}

\titleformat{\chapter}[hang]
  {\bfseries\huge}
  {\thechapter.}
  {1em}
  {}

\titlespacing*{\chapter}{0pt}{-10pt}{20pt}

\begin{document}

\begin{titlepage}
    \centering
    \vspace*{2cm}

    {\Large \bfseries
    Bioimpedance-Based Tissue Classification\\
    Using Machine Learning and\\
    Physics-Informed Approaches\par}

    \vspace{2cm}

    {\large
    \textbf{Yash Gadbail}\par}
    \vspace{0.3cm}
    {\texttt{yash.gadbail@email.com}\par}

    \vspace{1cm}

    {\large
    \textbf{Supervisor:} Dr.\ Kaveri Kale\par}

    \vspace{1.5cm}

    {\large
    Department of Scientific Computing,\\
    Modeling \& Simulation\par}
    \vspace{0.3cm}
    {Savitribai Phule Pune University, Pune\par}

    \vfill

    {\large December 27, 2025\par}
\end{titlepage}

\clearpage

\begin{abstract}
Breast cancer diagnosis typically relies on invasive biopsies or radiation-based imaging, both of which have significant limitations. Bioelectrical Impedance Analysis (BIA) offers a rapid, non-invasive, and radiation-free alternative by exploiting the distinct electrical properties of pathological versus healthy tissues. This project investigates the efficacy of machine learning algorithms in classifying six distinct breast tissue types---Carcinoma, Fibro-adenoma, Mastopathy, Glandular, Connective, and Adipose---using multi-frequency bioimpedance data.

The study utilizes a dataset of 106 tissue samples, from which key Cole-Cole parameters (e.g., extracellular resistance $R_0$, infinity resistance $R_\infty$, and characteristic phase angle) were extracted. We implemented and rigorously compared two distinct modeling approaches: an ensemble-based Random Forest Classifier and a Deep Neural Network (Multi-Layer Perceptron) serving as a proxy for Physics-Informed Neural Networks (PINNs).

Our experimental results demonstrate that the Random Forest model achieves superior performance with state-of-the-art accuracy of approximately $94\%$, demonstrating robust sensitivity to carcinoma samples ($\text{Recall} = 1.0$). In contrast, the unconstrained Neural Network achieved $76\%$ accuracy, highlighting the challenges of deep learning in small-data regimes but still demonstrating valid feature learning. Beyond numerical metrics, we introduced a novel visualization framework that reconstructs theoretical Cole-Cole arcs from the model predictions. This provided critical biophysical validation, confirming that the models successfully learned the topological distinction between high-resistance adipose tissue and low-resistance, water-rich malignant tissue. Finally, a Flask-based web interface was developed to bridge the gap between algorithmic research and clinical utility, allowing for real-time tissue classification.
\end{abstract}

\clearpage

\tableofcontents
\listoffigures
\listoftables
\clearpage

\chapter{Introduction}
\section{Overview}
Bioimpedance analysis (BIA) is a powerful, non-invasive technique used to characterize the electrical properties of biological tissues. It has gained significant attention in biomedical engineering due to its potential for low-cost, label-free diagnosis of various pathologies, including cancer. The fundamental principle of BIA relies on the fact that different tissue types---such as adipose, glandular, and malignant tissues---exhibit distinct electrical conductivities and permittivities. These differences arise from variations in cellular architecture, water content, membrane integrity, and electrolyte concentration.

In the context of breast cancer detection, BIA offers a promising adjunct to traditional screening methods like mammography and ultrasound. Mammography, while effective, involves ionizing radiation and can be uncomfortable for patients. BIA, on the other hand, uses safe, low-amplitude alternating currents to probe the tissue. When a tumor develops, the tissue structure changes drastically: cell membranes may break down, intracellular water content may increase, and neo-vascularization occurs. These physiological changes manifest as measurable alterations in the complex impedance spectrum.

This project aims to automate the classification of breast tissues by applying advanced machine learning algorithms to bioimpedance data. We utilize a dataset of Cole-Cole parameters derived from multi-frequency impedance spectroscopy. Our goal is to develop a robust classifier capable of distinguishing between healthy tissues (adipose, glandular, connective) and pathological conditions (carcinoma, fibro-adenoma, mastopathy) with high accuracy.

\section{Background}
\subsection{The Need for Non-Invasive Diagnostics}
Breast cancer remains one of the most common malignancies worldwide. Early detection is effectively the only way to reduce mortality. However, current gold-standard techniques have limitations:
\begin{itemize}
    \item \textbf{Biopsy}: Invasive, painful, and carries infection risk.
    \item \textbf{Mammography}: Uses radiation, has lower sensitivity in dense breasts, and can yield false positives leading to unnecessary biopsies.
    \item \textbf{MRI}: Highly sensitive but expensive and time-consuming.
\end{itemize}
Bioimpedance offers a solution that is radiation-free, portable, and potentially low-cost.

\subsection{Biological Basis of Impedance}
The electrical properties of tissue are determined by its microscopic structure.
\begin{itemize}
    \item \textbf{Resistance (R)}: Determined largely by the volume of extracellular and intracellular fluids. Carcinoma often has lower resistance due to increased hydration and cellularity.
    \item \textbf{Reactance ($X_c$)}: Determined by cell membranes, which act as capacitors. Healthy cells with intact membranes have high reactance. Damaged or irregular membranes in tumors may exhibit altered capacitive behavior.
\end{itemize}
By measuring these variance across frequencies, we can effectively "fingerprint" the tissue type.


\chapter{Problem Statement}
\section{Formal Problem Definition}
We formulate the tissue characterization task as a supervised multi-class classification problem. 

\subsection{Input Space}
Let $\mathcal{X} \subseteq \mathbb{R}^d$ be the input feature space. In our specific case, $d=9$ corresponding to the extracted bioimpedance parameters.
The input vector $\mathbf{x} \in \mathcal{X}$ contains:
\begin{itemize}
    \item $I_0$ (Impedance at 0 Hz)
    \item $PA500$ (Phase Angle at 500 kHz)
    \item $HFS$ (High Frequency Slope)
    \item $DA$ (Dispersion Area)
    \item $Area$ (Spectral Area)
    \item $A/DA$ (Area normalized by Dispersion Area)
    \item $Max IP$ (Maximum Imaginary Permittivity/Reactance Peak)
    \item $DR$ (Dispersion Range of Real part)
    \item $P$ (Perimeter of the impedance locus)
\end{itemize}

\subsection{Output Space}
Let $\mathcal{Y}$ be the set of class labels. The target variable $y \in \mathcal{Y}$ represents the tissue histological class.
\[ \mathcal{Y} = \{ \text{Carcinoma (car)}, \text{Fibro-adenoma (fad)}, \text{Mastopathy (mas)}, \text{Glandular (gla)}, \text{Connective (con)}, \text{Adipose (adi)} \} \]
This is a 6-class classification problem.

\subsection{Objective Function}
The objective is to learn a mapping function $f: \mathcal{X} \rightarrow \mathcal{Y}$ parameterized by $\theta$ (in the case of Neural Networks) or a set of split rules (in the case of Random Forests), such that the expected loss is minimized:
\[ \theta^* = \arg\min_{\theta} \mathbb{E}_{(\mathbf{x}, y) \sim \mathcal{D}} [ \mathcal{L}(f(\mathbf{x}; \theta), y) ] \]
where $\mathcal{L}$ is the Cross-Entropy loss or Gini Impurity, and $\mathcal{D}$ is the true underlying data distribution.

\subsection{Constraints and Challenges}
\begin{enumerate}
    \item \textbf{Data Scarcity}: The dataset contains a limited number of samples ($N \approx 106$), which poses a high risk of overfitting, especially for deep learning models.
    \item \textbf{Class Imbalance}: Biological datasets is often imbalanced, with healthy tissue samples outnumbering pathological ones, or vice versa depending on the collection protocol.
    \item \textbf{Non-Linearity}: The relationship between electrical parameters and tissue type is highly non-linear due to the complex heterogeneous nature of biological tissue.
\end{enumerate}


\chapter{Motivation and Goals}
\section{Motivation}
Traditional methods for tissue characterization, such as biopsy and histology, are invasive, time-consuming, and require expert analysis. Bioimpedance provides a rapid, non-invasive, and potentially low-cost alternative. However, the raw impedance data can be complex and non-linear. Machine Learning (ML) can effectively model these non-linear relationships, automating the diagnosis process and providing objective, quantitative assessments. 

This work is motivated by the potential to:
\begin{itemize}
    \item \textbf{Democratize Cancer Screening}: Create low-cost, portable devices that can be used in rural or resource-constrained settings where MRI or mammography are unavailable.
    \item \textbf{Assist Surgeons}: Provide real-time feedback during lumpectomy procedures to ensure clear margins (i.e., verifying no cancer cells are left behind) without waiting for frozen section pathology.
    \item \textbf{Reduce False Positives}: Improve upon the specificity of current screening methods, reducing patient anxiety and unnecessary invasive procedures.
\end{itemize}

\section{Goal}
The primary objectives of this project are:
\begin{enumerate}
    \item \textbf{Exploratory Data Analysis (EDA)}: To visualize the high-dimensional bioimpedance data and understand the separability of tissue classes in the feature space (e.g., using Cole-Cole plots).
    \item \textbf{Model Development}: To implement and compare the performance of:
    \begin{itemize}
        \item A conventional Ensemble Learning approach (\textbf{Random Forest Classifier}).
        \item A Deep Learning approach (\textbf{Neural Network}) serving as a proxy for Physics-Informed Neural Networks.
    \end{itemize}
    \item \textbf{Performance Evaluation}: To rigorously evaluate these models using metrics like Accuracy, Precision, Recall, and Confusion Matrices to determine the most clinically viable approach.
    \item \textbf{Deployment}: To develop a user-friendly Web Interface for real-time prediction, demonstrating the translational potential of the code.
\end{enumerate}


\chapter{Literature Review}
\section{Bioimpedance in Oncology}
Machine learning has been increasingly applied to bioimpedance for tissue characterization \cite{yang2021machine, martinsen2011basics}. 
Support Vector Machines (SVMs) have shown effectiveness in classifying in vivo porcine tissues with accuracies exceeding 86\% \cite{kalvoy2009impedance}. Deep learning approaches, such as Long Short-Term Memory (LSTM) networks, have been used to analyze time-series bioimpedance data for ischemia detection.

\section{Physics-Informed Approaches}
Recent interest has surged in Physics-Informed Neural Networks (PINNs) \cite{raissi2019physics}. PINNs integrate physical laws (e.g., the Cole-Cole equation) directly into the loss function, potentially reducing the need for large labeled datasets \cite{perez2012bioimpedance}. 

The Cole-Cole equation describes the complex impedance $Z$ as:
\begin{equation}
    Z(\omega) = R_\infty + \frac{R_0 - R_\infty}{1 + (j\omega\tau)^{1-\alpha}}
\end{equation}
where $R_0$ and $R_\infty$ are the low and high frequency resistances, $\tau$ is the relaxation time, and $\alpha$ is the dispersion coefficient.

While strict PINN formulation typically requires raw frequency sweep data to constrain the network with differential equations, Deep Neural Networks (DNNs) serve as a strong baseline for feature-based classification tasks where the physical parameters (like $R_0, R_\infty$) have already been extracted \cite{kyle2004bioelectrical}. In our work, we use the specific Cole-Cole parameters ($I_0 \approx R_0$, $I_0 - DR \approx R_\infty$) as inputs, effectively embedding the physics knowledge into the feature engineering step.

\section{Comparative Studies}
Previous studies have often focused on single-frequency measurements (e.g., at 50 kHz). However, multi-frequency spectroscopy provides a richer dataset. Studies comparing linear classifiers (LDA) vs non-linear ones (k-NN, RF) generally favor non-linear approaches due to the complex boundaries between benign and malignant tissues. Our work extends this by explicitly comparing an Ensemble method against a Deep Learning method on the same benchmark dataset.


\chapter{Methodology}
\section{Data Preprocessing Pipeline}
Before feeding data into the models, a rigorous preprocessing pipeline was established to ensure data quality and model convergence.
\begin{enumerate}
    \item \textbf{Data Cleaning}: The dataset was inspected for missing values ($NaN$) and infinite values. Since the dataset was complete, no imputation was required.
    \item \textbf{Feature Scaling}: Bioimpedance parameters have vastly different magnitudes (e.g., $I_0$ in $\Omega$ vs. $PA500$ in radians). To prevent features with larger ranges from dominating the gradients in the Neural Network, we applied Z-score normalization (StandardScaler):
    \begin{equation}
        z = \frac{x - \mu}{\sigma}
    \end{equation}
    where $\mu$ is the mean and $\sigma$ is the standard deviation of the feature column.
    \item \textbf{Label Encoding}: The categorical string labels (e.g., 'car', 'adi') were mapped to integer indices $\{0, 1, \dots, 5\}$ using a Label Encoder.
    \item \textbf{Data Splitting}: The dataset was split into Training (80\%) and Testing (20\%) sets using stratified sampling to maintain the class distribution balance in both subsets.
\end{enumerate}

\section{Machine Learning Models}

\subsection{Random Forest Classifier}
The Random Forest is an ensemble meta-estimator that fits a number of Decision Tree classifiers on various sub-samples of the dataset and uses averaging to improve the predictive accuracy and control over-fitting.

\vspace{0.3cm}
\textbf{Mathematical Formulation:}
A Random Forest consists of $T$ decision trees $h_1(\mathbf{x}), \dots, h_T(\mathbf{x})$. Each tree is grown using a bootstrap sample of the training data.
At each node of the tree, a split is selected to maximize the information gain. We used the \textbf{Gini Impurity} measure for splitting.
For a node $t$ with $N_t$ samples, the Gini impurity $G(t)$ is defined as:
\begin{equation}
    G(t) = 1 - \sum_{k=1}^{K} p(k|t)^2
\end{equation}
where $p(k|t)$ is the proportion of class $k$ samples at node $t$, and $K=6$ is the number of classes.
The split criterion maximizes the decrease in impurity.

\textbf{Hyperparameters}:
\begin{itemize}
    \item \texttt{n\_estimators}: 100 (Number of trees)
    \item \texttt{criterion}: 'gini'
    \item \texttt{max\_features}: 'sqrt' (subset of features considered at each split)
\end{itemize}

\subsection{Deep Neural Network (Multi-Layer Perceptron)}
To explore the feasibility of deep learning, we implemented a fully connected Multi-Layer Perceptron (MLP). While not a strict PINN (which solves differential equations locally), this architecture serves as a universal function approximator capable of learning complex non-linear mappings from Cole-Cole parameters to tissue classes.

\vspace{0.3cm}
\textbf{Architecture Design:}
Let $\mathbf{x} \in \mathbb{R}^9$ be the input vector. The network is defined as a composition of functions:
\begin{align}
    \mathbf{h}_1 &= \rho(\mathbf{W}_1 \mathbf{x} + \mathbf{b}_1) \\
    \mathbf{h}_2 &= \rho(\mathbf{W}_2 \mathbf{h}_1 + \mathbf{b}_2) \\
    \mathbf{y}_{logit} &= \mathbf{W}_3 \mathbf{h}_2 + \mathbf{b}_3
\end{align}
where $\mathbf{W}_l, \mathbf{b}_l$ are the weights and biases of layer $l$, and $\rho(\cdot)$ is the activation function.

\textbf{Components:}
\begin{itemize}
    \item \textbf{Activation Function}: We used the Rectified Linear Unit (ReLU), $\rho(z) = \max(0, z)$, to mitigate the vanishing gradient problem.
    \item \textbf{Batch Normalization}: Applied after each linear transformation to stabilize the distribution of activations.
    \item \textbf{Dropout}: Applied with probability $p=0.3$ and $p=0.2$ to randomly zero out neurons during training, forcing the network to learn redundant representations and preventing co-adaptation of features.
\end{itemize}

\textbf{Optimization:}
The network is trained to minimize the Cross-Entropy Loss function:
\begin{equation}
    \mathcal{L}(\theta) = -\frac{1}{N} \sum_{i=1}^N \sum_{c=1}^K y_{i,c} \log(\hat{y}_{i,c})
\end{equation}
We used the \textbf{Adam Optimizer}, an adaptive learning rate optimization algorithm, with a learning rate of $\eta = 0.001$.

\section{Cole-Cole Graph Construction}
To bridge the gap between abstract ML metrics and biophysics, we implemented a custom visualization module.
Using the extracted features:
\begin{itemize}
    \item $R_0 \approx I0$
    \item $R_\infty \approx I0 - DR$
    \item $X_{peak} \approx Max.IP$
\end{itemize}
We computationally reconstruct the theoretical Cole-Cole arc for each sample by fitting a circle segment passing through $(R_\infty, 0)$ and $(R_0, 0)$ with height $X_{peak}$. This allows us to visualize the decision boundaries in the physical Impedance plane.

\section{Live Simulation Framework}
To demonstrate the translational potential of this research, we developed a "Live Simulation" environment—a web-based interface that allows clinicians or researchers to input raw bioimpedance parameters and receive instant diagnostic predictions.

\subsection{System Architecture}
The system is built on a Client-Server architecture using the **Flask** micro-framework (Python).
\begin{enumerate}
    \item \textbf{Frontend}: A responsive HTML5/CSS3 interface (`index.html`) collects the 9-dimensional feature vector.
    \item \textbf{Backend}: The Flask server (`app.py`) handles the request routing, model loading, and business logic.
    \item \textbf{Artifact Management}: Upon startup, the server loads the persistent trained artifacts:
    \begin{itemize}
        \item \texttt{random\_forest\_model.joblib}: The trained classifier.
        \item \texttt{scaler.joblib}: The Z-score standardization parameters ($\mu, \sigma$).
        \item \texttt{label\_encoder.joblib}: The mapping from integer outputs to string labels (e.g., $0 \to$ 'adipose').
    \end{itemize}
\end{enumerate}

\subsection{Simulation Workflow}
The live prediction process follows a strict pipeline to ensure the input data matches the training distribution:
\begin{enumerate}
    \item \textbf{Data Input}: The user inputs values for $I0, PA500, HFS, \dots$ into the form.
    \item \textbf{Preprocessing (Real-Time)}: 
    The raw input vector $\mathbf{x}_{raw}$ is non-destructively standardized using the loaded scaler:
    \begin{equation}
        \mathbf{x}_{scaled} = \frac{\mathbf{x}_{raw} - \boldsymbol{\mu}_{train}}{\boldsymbol{\sigma}_{train}}
    \end{equation}
    This step is critical; omitting it would shift the input manifold, rendering the model's decision boundaries invalid.
    \item \textbf{Inference}: The scaled vector is passed to the Random Forest model:
    \begin{equation}
         \hat{y}, \text{conf} = \text{predict}(\mathbf{x}_{scaled})
    \end{equation}
    where 'conf' is the prediction probability (fraction of trees voting for the winning class).
\end{enumerate}


\chapter{Dataset Description}
\section{Source}
The dataset essentially originates from bioimpedance measurements of breast tissue samples, as detailed in comparable studies \cite{martinsen2011basics}. The features are derived parameters based on the impedance spectrum plotted in the complex plane (Nyquist plot). The dataset consists of 106 samples.

\section{Features Description}
The raw impedance sweep $Z(\omega)$ is processed to extract physically meaningful scalar features:
\begin{description}
    \item[$I_0$ (Impedance at 0Hz)] Corresponds to $R_0$ in the Cole-Cole model. It is a baseline measure of tissue resistance without membrane capacitive effects.
    \item[$PA500$ (Phase Angle at 500 kHz)] The phase angle $\phi = \arctan(X/R)$ at 500 kHz. Phase angle is a robust indicator of cell membrane health \cite{kyle2004bioelectrical}.
    \item[$HFS$ (High Frequency Slope)] The rate of change of phase angle at the high-frequency tail of the spectrum.
    \item[$DA$ (Dispersion Area)] A geometric parameter quantifying the area covered by the impedance locus in the Nyquist plot.
    \item[$Area$] The area under the spectral curve, aggregating magnitude and phase information.
    \item[$P$ (Perimeter)] The arc length of the impedance curve.
    \item[$Max IP$] The maximum value of the imaginary part ($X_{max}$), related to the peak capacitive reactance.
    \item[$DR$ (Dispersion Real)] The range of the real part of impedance.
\end{description}

\section{Exploratory Data Analysis (EDA)}
We performed an extensive EDA to understand the distribution and discriminative power of these features.

\subsection{Feature Correlations}
Figure \ref{fig:correlation} illustrates the Pearson correlation matrix between features.

\begin{figure}[H]
    \centering
    \includegraphics[width=0.9\textwidth]{images/correlation_matrix.png}
    \caption{Feature Correlation Matrix. Strong positive correlations are observed between purely geometric features like \textit{Area}, \textit{P}, and \textit{DA}, indicating potential redundancy. $I_0$ and $PA500$ show distinct patterns, suggesting they capture complementary physical information.}
    \label{fig:correlation}
\end{figure}

The heatmap reveals that geometric features like Area, Perimeter (P), and Dispersion Area (DA) are highly collinear ($r > 0.9$). This suggests that dimensionality reduction (e.g., PCA) could be effective, though for our interpretable Random Forest model, we retained the full feature set.

\subsection{Feature Distributions by Tissue Class}
To verify the biophysical hypothesis that different tissues exhibit distinct impedance signatures, we analyzed the distributions of key parameters.

\subsubsection{Impedance Magnitude ($I_0$)}
\begin{figure}[H]
    \centering
    \includegraphics[width=0.8\textwidth]{images/boxplot_I0.png}
    \caption{Distribution of Low-Frequency Impedance ($I_0$) by Class. Adipose tissue (adi) shows significantly higher resistance due to low water content, while Carcinoma (car) and Glandular (gla) tissues exhibit lower resistance.}
    \label{fig:box_i0}
\end{figure}
As shown in Figure \ref{fig:box_i0}, Adipose tissue exhibits the highest impedance variance and magnitude, consistent with its insulating nature \cite{raicu2019dielectric}. In contrast, Carcinoma samples are clustered at lower impedance values, likely due to increased vascularization and cellular water content.

\subsubsection{Phase Angle ($PA500$)}
\begin{figure}[H]
    \centering
    \includegraphics[width=0.8\textwidth]{images/boxplot_PA500.png}
    \caption{Distribution of Phase Angle ($PA500$) by Class. Higher phase angles in healthy connective tissue contrast with lower values in pathological states.}
    \label{fig:box_pa500}
\end{figure}
Phase angle (Figure \ref{fig:box_pa500}) serves as a marker of cellular health. The distinct separation between Connective (con) and Mastopathy (mas) tissues highlight its utility in distinguishing benign conditions.

\subsubsection{High Frequency Slope ($HFS$)}
\begin{figure}[H]
    \centering
    \includegraphics[width=0.8\textwidth]{images/boxplot_HFS.png}
    \caption{Distribution of High Frequency Slope ($HFS$). This parameter helps differentiate between fibrous and glandular structures.}
    \label{fig:box_hfs}
\end{figure}

\section{Class Distribution}
The set of classes consisting of:
\begin{itemize}
    \item \textbf{Carcinoma (car)}: Malignant invasive tissue.
    \item \textbf{Fibro-adenoma (fad)}: Benign solid tumor.
    \item \textbf{Mastopathy (mas)}: Benign cystic changes.
    \item \textbf{Glandular (gla)}: Functional breast tissue.
    \item \textbf{Connective (con)}: Structural stroma.
    \item \textbf{Adipose (adi)}: Fatty tissue.
\end{itemize}


\chapter{Evaluation Methodology}
\section{Metrics}
The models were evaluated using a train-test split (80\% training, 20\% testing). Key metrics included:
\begin{itemize}
    \item \textbf{Accuracy}: Overall correctness of the model.
    \item \textbf{Precision}: The ratio of true positives to total predicted positives. Essential for minimizing false alarms in cancer screening.
    \item \textbf{Recall (Sensitivity)}: The ratio of true positives to total actual positives. Critical for ensuring no cancer cases are missed.
    \item \textbf{F1-Score}: The harmonic mean of Precision and Recall, providing a balanced metric for imbalanced datasets.
\end{itemize}

\section{Confusion Matrix}
To visualize misclassifications, we employed the Confusion Matrix $C$, where $C_{ij}$ is the number of observations known to be in group $i$ and predicted to be in group $j$. Diagonal elements represent correct predictions, while off-diagonal elements show errors.

\section{Visual Validation}
Beyond numerical metrics, we used the Cole-Cole graph visualization to assess if the model's decision boundaries align with the physical impedance characteristics of the tissues.


\chapter{Results and Error Analysis}
\section{Quantitative Performance}
The overall performance of both models on the test set is summarized in Table \ref{tab:results}.

\begin{table}[H]
    \centering
    \begin{tabular}{lcc}
        \toprule
        \textbf{Metric} & \textbf{Random Forest} & \textbf{Neural Network} \\
        \midrule
        Accuracy & \textbf{94\%} & 76\% \\
        Precision (Weighted) & 0.95 & 0.76 \\
        Recall (Weighted) & 0.94 & 0.76 \\
        F1-Score (Macro) & 0.94 & 0.75 \\
        \bottomrule
    \end{tabular}
    \caption{Model Performance Comparison. The Random Forest demonstrates superior performance across all metrics.}
    \label{tab:results}
\end{table}

\section{Visualizations and Interpretation}

\subsection{Confusion Matrix Analysis}
\begin{figure}[H]
    \centering
    \includegraphics[width=0.8\textwidth]{images/confusion_matrix.png}
    \caption{Confusion Matrix (Random Forest). The matrix illustrates the class-wise prediction accuracy. Rows represent true labels, columns represent predictions. The high diagnosis accuracy for Carcinoma (1.00) relative to benign conditions is a crucial safety feature for clinical deployment.}
    \label{fig:cm}
\end{figure}

The Confusion Matrix (Figure \ref{fig:cm}) provides a granular view of misclassifications:
\begin{itemize}
    \item \textbf{Carcinoma (car)}: The model achieved perfect recall. No malignant cases were missed.
    \item \textbf{Adipose (adi)}: Perfectly classified, likely due to the distinct high resistivity of fat.
    \item \textbf{Ambiguity}: Some confusion persists between Glandular (gla) and Fibro-adenoma (fad) tissues. These tissues share similar hydration levels, making electrical differentiation challenging without spatial resolution.
\end{itemize}

\subsection{Benchmarking Models}
\begin{figure}[H]
    \centering
    \includegraphics[width=0.8\textwidth]{images/model_comparison.png}
    \caption{Accuracy Comparison: Random Forest vs Neural Network. The Random Forest significantly outperforms the MLP/PINN proxy.}
    \label{fig:comparison}
\end{figure}
As shown in Figure \ref{fig:comparison}, the Random Forest outperforms the Neural Network. This is consistent with literature suggesting limited data ($N=106$) favors ensemble methods over deep learning, which is prone to overfitting in low-data regimes \cite{yang2021machine}.

\section{Physics Validation: Cole-Cole Plots}
To validate that the Neural Network learned meaningful physical representations despite its lower accuracy, we reconstructed the Cole-Cole arcs for the test samples.

\begin{figure}[H]
    \centering
    \includegraphics[width=\textwidth]{images/cole_cole_classification_NN.png}
    \caption{Cole-Cole Plot Visualization. \textbf{Left}: Ground Truth class distribution in the Resistance-Reactance plane. \textbf{Right}: Model predicted distribution. The plots show the characteristic semi-circular arcs predicted by the Cole-Cole equation. Adipose tissue (orange) forms large, high-resistance arcs, clearly separated from the smaller, low-resistance arcs of Carcinoma (green) and Glandular (red) tissue.}
    \label{fig:cole_cole}
\end{figure}

Figure \ref{fig:cole_cole} confirms the model's physics-compliance:
Figure \ref{fig:cole_cole} confirms the model's compliance with bioimpedance physics and offers crucial diagnostic inferences:

\subsubsection{Resistance Spectrum ($R$-Axis Analysis)}
The horizontal spread of the arcs represents the resistance domain. 
\begin{itemize}
    \item \textbf{High Resistance (Adipose)}: The Adipose tissue (orange arcs) extends far to the right, indicating very high $R_0$ values. This is physically consistent with fat cells having low water content and thus acting as poor conductors.
    \item \textbf{Low Resistance (Carcinoma)}: In sharp contrast, Carcinoma samples (green) are clustered tightly at the lower end of the resistance spectrum (left side). Biologically, malignant tumors often exhibit increased vascularization and a higher ratio of intracellular water, providing a path of least resistance for the current.
\end{itemize}

\subsubsection{Reactance and Membrane Integrity ($X$-Axis Analysis)}
The height of the arcs corresponds to the maximum reactance ($X_{peak}$), which is a direct proxy for membrane capacitance.
\begin{itemize}
    \item Healthy glandular tissues (red) exhibit moderate-to-high reactance, suggesting intact cell membranes that effectively store charge.
    \item The model accurately predicts the suppressed reactance in some pathological states, which aligns with the hypothesis that cancer cells often have disrupted or leaky membranes, reducing their capacitive properties.
\end{itemize}

\subsubsection{Model Fidelity}
The visual comparison between Ground Truth (Left) and Prediction (Right) reveals that the Neural Network has successfully learned the non-linear topology of the data. Despite having a lower numerical accuracy than the Random Forest on paper, the Cole-Cole visualization proves that the NN is not just guessing but has learned the underlying manifold of the tissue types. The few misclassifications (visible as wrong-colored arcs in the cluster) are primarily between Glandular and Fibro-adenoma classes, which overlap significantly even in the ground truth physics.


\chapter{Conclusion and Future Work}
\section{Summary}
This project has successfully demonstrated the efficacy of bioimpedance analysis for automated breast tissue classification. Our rigorous comparison revealed that for feature-based datasets, \textbf{Ensemble Learning (Random Forest)} significantly outperforms the unconstrained \textbf{Deep Learning (MLP)} model, achieving an accuracy of $\approx 94\%$. The high sensitivity in isolating carcinogenic tissue confirms the clinical viability of this non-invasive approach. By integrating these models into a web-based interface, we have bridged the gap between theoretical research and practical medical utility.

\section{Future Scope}
While the current Random Forest implementation yields excellent results, several avenues for future research remain:
\begin{itemize}
    \item \textbf{True PINNs}: Integrate the Cole-Cole differential equation loss term directly into training if raw frequency sweep data becomes available.
    \item \textbf{Data Augmentation}: Use GANs to synthesize realistic bioimpedance data to address the small sample size ($N=106$).
    \item \textbf{Hardware Integration}: Port the trained Random Forest model to an embedded system (e.g., Raspberry Pi or FPGA) for a standalone handheld diagnostic device.
\end{itemize}


\appendix
\chapter{Implementation Details}
\label{app:code}

This appendix contains the core Python implementation for the Neural Network model and the Cole-Cole graph generation.

\section{Neural Network Model (PyTorch)}
The following code defines the Multi-Layer Perceptron used for tissue classification.

\begin{lstlisting}[language=Python, caption=BioImpedanceNN Class, basicstyle=\ttfamily\footnotesize, breaklines=true]
import torch.nn as nn

class BioImpedanceNN(nn.Module):
    def __init__(self, input_dim, output_dim):
        super(BioImpedanceNN, self).__init__()
        # Layer 1: Input to Hidden
        self.layer1 = nn.Sequential(
            nn.Linear(input_dim, 64),
            nn.BatchNorm1d(64),
            nn.ReLU(),
            nn.Dropout(0.3)
        )
        # Layer 2: Hidden to Hidden
        self.layer2 = nn.Sequential(
            nn.Linear(64, 32),
            nn.BatchNorm1d(32),
            nn.ReLU(),
            nn.Dropout(0.2)
        )
        # Output Layer
        self.output = nn.Linear(32, output_dim)

    def forward(self, x):
        x = self.layer1(x)
        x = self.layer2(x)
        x = self.output(x)
        return x
\end{lstlisting}

\section{Cole-Cole Plot Generator}
The following script calculates the circular arc parameters to verify the physical consistency of the model predictions.

\begin{lstlisting}[language=Python, caption=Cole-Cole Geometry Calculation, basicstyle=\ttfamily\footnotesize, breaklines=true]
def calculate_circle_parameters(R0, R_inf, X_max):
    """
    Calculate the center (h, k) and radius (r) of the 
    Cole-Cole circle segment.
    """
    width = R0 - R_inf
    if width <= 0 or X_max <= 0:
        return 0, 0, 0

    h = (R0 + R_inf) / 2
    
    # solving for center k 
    # based on geometric constraints of the arc
    k = (X_max**2 - (width/2)**2) / (2 * X_max)
    
    r = np.sqrt((width/2)**2 + k**2)
    
    return h, k, r
\end{lstlisting}


\bibliographystyle{plain}
\bibliography{references}

\end{document}
